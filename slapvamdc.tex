\documentclass[11pt,a4paper]{ivoa}
\input tthdefs
\usepackage{float}


\title{Mapping VAMDC spectral line data to SLAP}

% see ivoatexDoc for what group names to use here
\ivoagroup{???? group ????}

\author{Margarida Castro Neves}

\editor{????Alfred Usher Thor????}

% \previousversion[????URL????]{????Funny Label????}
\previousversion{This is the first public release}
       

\begin{document}
\begin{abstract}
VAMDC contains  a great amount of spectral line data, that could be made more
accessible from VO Applications with  VO protocols. This document proposes a mapping
between VAMDC metadata and SSLDM (Simple Spectral Line Data Model) metadata.
\end{abstract}


\section*{Acknowledgments}

???? Or remove the section header ????

\section*{Conformance-related definitions}

The words ``MUST'', ``SHALL'', ``SHOULD'', ``MAY'', ``RECOMMENDED'', and
``OPTIONAL'' (in upper or lower case) used in this document are to be
interpreted as described in IETF standard RFC2119 \citep{std:RFC2119}.

The \emph{Virtual Observatory (VO)} is a
general term for a collection of federated resources that can be used
to conduct astronomical research, education, and outreach.
The \href{http://www.ivoa.net}{International
Virtual Observatory Alliance (IVOA)} is a global
collaboration of separately funded projects to develop standards and
infrastructure that enable VO applications.


\section{Introduction}

SLAP (Simple Line Access Protocol) is the VO standard for Spectral Line querying. 
SSLDM (Simple Spectral Line Access Protocol) is the VO data model for spectral lines.
There are very few working services in the VO that use this protocols, meaning that the
amount of spectral line data in the VO is small. 
On the other side, VAMDC services offer a great amount of spectral line data, and 
so it's desirable to access this data in a VO-way (simple). 
The queries to the VAMDC services might be simple, however, the data returned is much more
than the necessary data for simple use.
A proposed mapping would translate the VAMDC output into the  VO data model (SSLDM). Some metadata that are not contained in SSLDM can also be retrieved, if useful. 
Which parameters  will be mapped are defined by the use cases.

\section{Use Cases}

In most use cases, the goal is to search for spectral lines within a wavelength range:

\begin{itemize}
\item all spectral lines 
\item spectral lines from a specific atom (or molecule)
\item spectral lines from a specific ionisation level
\item spectral lines with specific (other) parameter

\end{itemize}

\section{Service discovery and Queries}
TO DO

discovery -> VAMDC registry queries
querying ->  SLAP queries / VAMDC queries
What is done in the Client, what is done in the server, how it would work
How could slap clients use vamdc data

\section{Mapping}

\subsection{First mapping SSLDM - VAMDC}
For SPLAT-VO, a mapping between SSLDM and VAMDC data model has been defined and implemented:

\begin{table}[H]
\small
%\begin{center}
\begin{tabular}{ |p{0.2\linewidth}  |p{0.4\linewidth} | p{0.4\linewidth}| }
\hline

Name & SSLDM Utype & VAMDC \\
\hline
title  &  (ssldm:)line.title  & ---  \\
element  & line.species  & ElementSymbol  \\
wavelength  & line.wavelength.value &  EnergyWavelength  \\
air wavelength & line.airWavelength.value  & EnergyWavelength  \\
initial energy  & line.initialLevel.energy.value & (AtomicState) IonizationEnergy  \\
final energy &  line.finalLevel.energy.value  & (AtomicState) IonizationEnergy  \\
ionisation level & ---  & IonCharge  \\
einsteinA & line.einsteinA.value  & TransitionProbabilityA \\
initial level & line.initialLevel  &  (AtomicState) Description \\
final level & line.finalalLevel  &  (AtomicState) Description \\
oscillator strength & line.oscillatorStrength & ProbabilityOscillatorStrength  \\
weighted oscillator strength & line.weightedOscillatorStrength & ProbabilityWeightedOscillator- Strength \\
\hline
\end{tabular}
%}
%\end{center}
%\label{default}
\end{table}


\normalsize

\subsection{Proposed parameters to me mapped}

Relevant parameters have to be chosen based on IVOA use cases. Listed below are selected elements from the XSAMSVAMDC Data Model. \textit{XSAMSData.Species.Atoms.Atom} (molecules are not considered in this first version) and
\textit{XSAMSData.Processes.Radiative.RadiativeTransition} have been used.



\begin{description}

	
\item [Wavelength] (in vacuum)\hfill\\
	XSAMSData.Processes.Radiative.RadiativeTransition.EnergyWavelength.Wavelength\\
	(if XSAMSData.Processes.Radiative.RadiativeTransition.EnergyWavelength.Wavelength.Vacuum = false,
	then the the value in \\XSAMSData.Processes.Radiative.RadiativeTransition.EnergyWavelength.Wavelength.AirToVacuum 
	is used).
	
\item [ElementSymbol]\hfill\\
	Species.Atoms.Atom.ChemicalElement.ElementSymbol\\
	(Species where XSAMSData.Processes.Radiative.RadiativeTransition.SpeciesRef equals to 
	Species.Atoms.Atom.Isotope.Ion.SpeciesID)
	
\item [IonCharge]\hfill\\
	Species.Atoms.Atom.Isotope.Ion.IonCharge\\
	(Species where XSAMSData.Processes.Radiative.RadiativeTransition.SpeciesRef equals to 
	Species.Atoms.Atom.Isotope.Ion.SpeciesID )


\item [Title]\hfill\\
	can be composed from element symbol and ion charge
\end{description}


\section{Server-side or Application-side translation}


\appendix
\section{Changes from Previous Versions}

No previous versions yet.  
% these would be subsections "Changes from v. WD-..."
% Use itemize environments.


\bibliography{ivoatex/ivoabib,ivoatex/docrepo}


\end{document}
