\documentclass[11pt,a4paper]{ivoa}
\input tthdefs
\usepackage{float}
\usepackage{todonotes}

\lstloadlanguages{SQL,XML}
\lstset{flexiblecolumns=true,numberstyle=\small,showstringspaces=False,
  identifierstyle=\texttt}

\title{IVOA Relational Model for Spectral Lines (LineTAP)}

% see ivoatexDoc for what group names to use here
\ivoagroup{DAL}

\author{Castro Neves, M.}
\author{Moreau, N.}
\author{Demleitner, M.}

\editor{Margarida Castro Neves, Nicolas Moreau}

% \previousversion[????URL????]{????Funny Label????}
\previousversion{This is the first public release}

\begin{document}
\begin{abstract}

This document proposes a schema for a database table containing
descriptions of spectral lines.  Together with the IVOA's Table Access
Protocol TAP, this results in a powerful facility for spectral line
discovery and identification.  The underlying model is rooted in the
widely-depolyed VAMDC, and the intent is that at least the atomic and
molecular data from VAMDC can easily be re-published using LineTAP.

\end{abstract}


%\section*{Acknowledgments}


\section*{Conformance-related definitions}

The words ``MUST'', ``SHALL'', ``SHOULD'', ``MAY'', ``RECOMMENDED'', and
``OPTIONAL'' (in upper or lower case) used in this document are to be
interpreted as described in IETF standard RFC2119 \citep{std:RFC2119}.

The \emph{Virtual Observatory (VO)} is a
general term for a collection of federated resources that can be used
to conduct astronomical research, education, and outreach.
The \href{http://www.ivoa.net}{International
Virtual Observatory Alliance (IVOA)} is a global
collaboration of separately funded projects to develop standards and
infrastructure that enable VO applications.


\section{Introduction}

The Simple Line Access Protocol SLAP \citep{2010ivoa.specQ1209O}
currently is the VO
recommendation for querying spectral line collections. 
It is based on the Simple Spectral Line Data
Model SSLDM \citep{2010ivoa.spec.1209O}, which defines the underlying
data model.
More than ten years after the protocol's definition, there are still
very few SLAP services registred in the VO.

On the other hand, the Virtual Atomic and Molecular Data Center
VAMDC\todo{reference?} offers a great amount of spectral line
data.  Making this data available to VO clients without major extra
tooling is certainly desirable.  While the query part of VAMDC clearly
betrays its origins in VO standards and thus might readily be integrated
into the VO protocol stack, the service output
comes in a very comprehensive derivative of the XML Schema
for Atomic, Molecular and Solid Data XSAMS \citep{XSAMS:Docs}; in
particular, its tree-like nature complicates casual use.  In addition, many
interesting use cases can already be satisfied with a simple relational
mapping of XSAMS. \todo{Prior art: HTML table XSLT from VAMDC -- do we
want to reference it?}

This document proposes LineTAP, a simple way to access spectral line
data through a VO service employing such a simplified relational 
mapping.  We give this mapping from the VAMDC-XSAMS Data 
Model to the
selected set of line quantities of great relevance to astronomy in
section~\ref{sect:sld}.

When accessed using the Table Access Protocol TAP \citep{std:TAP}, the
table can be queried using the expressive SQL-derived query language
ADQL, while query results are available in the VOTable format, easily
readable by VO client applications. 




\section{Use Cases and Requirements}

LineTAP really only has a single use case, the discovery of spectral
lines for identification purposes.  To structure standards development,
we discuss some situations specifically:

\subsection{Identifying a Single Line}

A user sees a feature in a spectral with known (and realiable) spectral
calibration and now wants to know what might possibly be responsible for
it.  Hence, they query a narrow spectral range and retrieve all known
lines from all services.

To select which of the candidate lines are plausible matches, users
would inspect line metadata such as the originating atom or element, the
ionisation state, and perhaps oscillator strengths.


\subsection{Getting Properties of Well-Known Lines}

A user wants to display, say, the Lyman series over a plot of a
spectrum.  Hence, a client needs to discover which service holds such
data, select the appropriate records -- presumably by their properties,
perhaps even by their name --, and retrieve them.  If multiple services
hold the desired data, it might need to reconcile differing
specifications.


\subsection{Retrieving Lines For Ploting}

A client program displays a spectrum and wants to mark location of
spectral features that might exist in the source object.

The challenge here is that displaying all lines known obviously is
impossible and would not help users in any way.  Hence, the client needs
to have some idea of which lines can be expected to be strong given the
physics of the emission's source region.

Even retrieving the lines before selection can be optimised greatly in
this case, as in wider spectra at least hundreds of thousands of lines
will be within the spectral range, while it probably rarely makes sense
to plot more than a hundred or so.  Hence, careful selection of lines
can reduce the volume of data transferred and processed by the client by
large factors.

To make good on this promise, the tables need to be queriable such that
lines suspected to be strong for some combination of chemistry,
temperature, and pressure can be filtered out with some accuracy.


\subsection{Molecular Chemistry}

A specific challenge when specifying chemistry are molecules because
they are so much harder to characterise than atoms.  While InChi and
InChi keys help here, user acceptance of them at this point is at least
questionable.  As a fallback mechanism, and perhaps also as a means of
characterising servers, users may want to discover all species served by
a specific server.


\subsection{Credit}

In particular to provide an incentive to contribute to the global
repository of line data, it should be as simple as possible for users to
give credit to the contributors of line data.


\subsection{Non-Use Cases}

This specification differs from VAMDC in that it does not attempt to
cover all possible uses of spectral line data.  In particular, no
attempt is made to

\begin{itemize}
\item Publish sufficient information to feed sophisticated,
high-precision atmosphere models.
\item Deal with solid-state spectroscopy.
\item Publish spectra of non-electromagnetic messengers.
\end{itemize}



\section{Protocol, Service discovery and queries - (need better title)}

\subsection{ Queries: LineTAP }
\todo{ description, examples, translation VAMDC Queries}


\section{Spectral Line Data}\label{quantities}
\label{sect:sld}

We define the following atomic quantities based on IVOA use cases:

\todo{ method: define other method options besides experiment and theory?}
\todo{ add more information to this table (data type, mandatory/optional, etc)}
\begin{table}[H]
%\small
\begin{center}
\begin{tabular}{l p{8cm}}%{l l}% p{0.4\linewidth}}
\texttt{title} & the title of the line, a human readable string representing the species originating  the line.  \\
\texttt{vacuum\_wavelength} & wavelength in vacuum, in Angstrom \\
\texttt{vacuum\_wavelength\_error} & error for the measured vacuum wavelength \\
\texttt{method} & method the wavelength was obtained with: \textit{experiment} (observed wavelength), or \textit{theory} (wavelength calculated from theoretical models) \\ 
\texttt{stoichiometric\_formula} & string describing the chemical element (symbol) or the molecule (formula) \\
\texttt{ion\_charge} & ionisation level\\
\texttt{atom\_mass\_number} & atom mass number\\
\texttt{upper\_state\_configuration} & upper state configuration\\
\texttt{lower\_state\_configuration} & lower state configuration\\
\texttt{upper\_energy} & energy of the upper state \\
\texttt{lower\_energy} & energy of the lower state\\
\texttt{inchi} & chemical species inchi \\
\texttt{inchikey} & chemical species inchikey \\
\texttt{einstein\_a} & Einstein A coefficient\\
\texttt{oscillator\_strength} & oscillator strength of radiative transition \\
\texttt{weighted\_oscillator\_strength} &  Weighted oscillator strength of radiative transition \\
\texttt{line\_strength} & Total absorption by a spectra line \\
\texttt{line\_reference\_doi} & Digital Object Identifier of bibliography source \\
\texttt{line\_reference\_uri} & Web link to the publication\\
\end{tabular}

\end{center}
\end{table}




%Returnables corresponding to a "DataType" in the XSAMS schema (numerical values) have additional categories, besides the value. Following categories will be also used in the mapping, \textit{Keyword} being the returnable keyword:

% \begin{itemize}
%\item \textit{Keyword}Unit  - the units of the value. If the wavelength unit returned is \textbf{A}, it must be corrected to \textbf{Angstrom} for compatibility
%\item \textit{Keyword}Method - the method used to obtain this value (e.g. experiment, theory, etc)
%\end{itemize}

\section{LineTAP use case examples}


\section{Mapping from VAMDCXSAMS}

The quantities used in the mapping belong to the following branches of VAMDCXSAMS:\\\\
\textit{XSAMSData.Species.Atoms.Atom}  referred as ATOM,\\
\textit{XSAMSData.Species.Atoms.AtomicState}  referred as ATOMICSTATE,\\
\textit{XSAMSData.Species.Molecules.Molecule}, referred as MOLECULE,\\
\textit{XSAMSData.Processes.Radiative.RadiativeTransition}, referred as RADTRANS.\\
\textit{XSAMSData.Sources}, referred as RADTRANS.\\


The VAMDC-TAP query language defines a list of keywords, called
\href{https://standards.vamdc.eu/dictionary/returnables.html}{Returnables},
listing the quantities that can be returned by a service.They correspond
to an element in the XSAMS data model. The returnables supported by a
service are declared by a service manager when he deploys the software
on its database, and saved in the VAMDC registry. \\

Listed below are the mappings for the atomic quantities defined in
\ref{quantities} (at the moment only atoms, not molecules):

\renewcommand{\descriptionlabel}[1]{\hspace{\labelsep}\texttt{#1}}
\begin{description}

\item [vacuum\_wavelength] (in Angstrom)\hfill\\
    \textit{data model:} RADTRANS.EnergyWavelength\\
	\textit{returnables needed:} RadTransWavelength, RadTransWavelength.Vacuum, RadTransWavelengthAirToVac, RadTransWavelengthUnit\\
	\textit{constraints:} RadTransWavelengthVacuum = true; else use RadTransWavelengthAirToVac.\\
	
\item [vacuum\_wavelength\_error]  \hfill\\\todo{Wavelength error? How to get it from VAMDC data?}

    \textit{data model:} RADTRANS.EnergyWavelength\\
	\textit{returnables needed:} RadTransWavelength, RadTransWavelength.Vacuum, RadTransWavelengthAirToVac, RadTransWavelengthUnit\\
	\textit{constraints:} RadTransWavelengthVacuum = true; else use RadTransWavelengthAirToVac.\\
	
\item [method] \hfill\\
	\textit{data model:} Methods.Method.Category\\
	\textit{returnables needed:} RadTransWavelengthMethod\\
	\textit{constraints:} Methods.Method.MethodID=RadTransWavelength.Wavelength@methodRef\\

\item [inchi] \hfill\\
	\textit{data model:} ATOM.Isotope.Ion.InChi,  MOLECULE.MolecularChemicalSpecies.InChI\\
        \textit{returnables needed:} AtomInchi, MoleculeInChi\\
	\textit{constraints:} RadTrans.SpeciesRef = Atom.SpeciesID or MoleculeSpeciesID
	
\item [inchikey] \hfill\\
	\textit{data model:} ATOM.Isotope.Ion.InChiKey, MOLECULE.MolecularChemicalSpecies.InChIKey \\
         \textit{returnables needed:} AtomInchiKey, MoleculeInChiKey\\
         \textit{constraints:} RadTrans.SpeciesRef = Atom.SpeciesID or  Molecule.SpeciesID

\item [stoichiometric\_formula] \hfill\\
	\textit{data model:} ATOM.ChemicalElement.ElementSymbol,\\ MOLECULE. MolecularChemicalSpecies.StoichiometricFormula\\
	\textit{returnables needed:} RadTrans.SpeciesRef, Atom.SpeciesID, Molecule.SpeciesID, Atom.Symbol, Molecule.MoleculeStoichiometricFormula\\
	\textit{constraints:}  RadTrans.SpeciesRef = Atom.SpeciesID or  Molecule.SpeciesID

\item [ion\_charge]\hfill\\
	\textit{data model:} ATOM.Isotope.Ion.IonCharge or MOLECULE.MolecularChemicalSpecies.IonCharge \\
	\textit{returnables needed:} RadTransSpeciesRef, AtomSpeciesID, AtomIonCharge, MoleculeSpeciesID, MoleculeIonCharge \\
	\textit{constraints:}RadTransSpeciesRef=AtomSpeciesID or  Molecule.SpeciesID

	\item [atom\_mass\_number]\hfill\\
	\textit{data model:} ATOM.Isotope.IsotopParameters.MassNumber \\
	\textit{returnables needed:} AtomMassNumber\\
	\textit{constraints:}RadTransSpeciesRef=AtomSpeciesID

	\item [upper\_state\_configuration]\hfill\\
	\textit{data model:} ATOMICSTATE.Description, MOLECULARSTATE.Description\\
	\textit{returnables needed:} RadTransUpperStateRef, AtomStateID, MoleculeStateID,\\ AtomStateDescription, MoleculeStateDescription\\
	\textit{constraints:}  RadTransUpperStateRef = AtomStateID

	\item [lower\_state\_configuration]\hfill\\
	\textit{data model:} ATOMICSTATE.Description, MOLECULARSTATE.Description\\
	\textit{returnables needed:} RadTransUpperStateRef, AtomStateID, MoleculeStateID,\\ AtomStateDescription, MoleculeStateDescription\\
	\textit{constraints:}  RadTransUpperStateRef = AtomStateID or MolecularStateID
	
	\item [upper\_state\_energy]\hfill\\
	\textit{data model:} ATOMICSTATE.AtomicNumericalData.StateEnergy, \\ MOLECULARSTATE.MolecularStateCharacterisation.StateEnergy\\
	\textit{returnables needed:} RadTransUpperStateRef, AtomStateID, MoleculeStateID\\
	\textit{constraints:}  RadTransUpperStateRef = AtomStateID or MoleculeStateID
	
	\item [lower\_state\_energy]\hfill\\
	\textit{data model:} ATOMICSTATE.AtomicNumericalData.StateEnergy, \\ MOLECULARSTATE.MolecularStateCharacterisation.StateEnergy\\
	\textit{returnables needed:} RadTransLowerStateRef, AtomStateID,  MoleculeStateID\\
	\textit{constraints:}  RadTransLowerStateRef = AtomStateID or MoleculeStateID

	\item [einstein\_a]\hfill\\
	\textit{data model:}  RADTRANS.Probability.TransitionProbabilityA\\
	\textit{returnables needed:} RadTransProbabilityA\\
	\textit{constraints:}

	\item [oscillator\_strength]\hfill\\
	\textit{data model:}  RADTRANS.Probability.TransitionOscillatorStrength\\
	\textit{returnables needed:} RadTransProbabilityOscillatorStrength\\
        \textit{constraints:}

	\item [weighted\_oscillator\_strength]\hfill\\
	\textit{data model:}  RADTRANS.Probability.TransitionWeightedOscillatorStrength\\
	\textit{returnables needed:} RadTransProbabilityWeightedOscillatorStrength\\
 	   \textit{constraints:}

	\item [line\_reference\_doi]\hfill\\
	\textit{data model:} SOURCES.Source.DigitalObjectIdentifier\\
	\textit{returnables needed:} SourceDOI\\
	\textit{constraints:}

	\item [line\_reference\_uri]\hfill\\
	\textit{data model:} SOURCES.Source.UniformResourceIdentifier\\
	\textit{returnables needed:} SourceURI\\
	\textit{constraints:}

	\item [line\_strength]\hfill\\
	\textit{data model:} RadTrans.Probability.TransitionLineStrength\\
	\textit{returnables needed:} RadTransProbabilityLineStrength\\
        \textit{constraints:}

\item [title]\hfill\\
	a string composed by the values of  \texttt{element\_symbol} and \texttt{ion\_charge}


\end{description}

\section{LineTAP and the VO Registry}

\subsection{Registering LineTAP-conforming Tables}

LineTAP tables are registered using VODataService
\citep{2010ivoa.spec.1202P} tablesets, where the table utype is set to
$$\hbox{\verb|ivo://ivoa.net/std/linetap#table-1.0|}.$$

The tableset is normally contained in a VODataService \xmlel{CatalogService}
record with a TAP capability, and this capability normally is an auxiliary
capability as per DDC \citep{2019ivoa.rept.0520D}.  For one-table
services a full TAPRegExt \citep{2012ivoa.spec.0827D} capability is also
allowed; other resource types can be used for registration as
appropriate.

Further capabilities, for instance for full VAMDC or legacy SLAP
services, may be given in the same record.

An example for a registry record in VOResource, for the case of
using an auxiliary capability referencing a main TAP service comes with
this document\footnote{\auxiliaryurl{example-record.xml}}.

The noteworthy points in the record are:

\begin{itemize}
\item A \xmlel{relationship} element referencing the main TAP service 
through which the service is queriable as per DDC:
\begin{lstlisting}[language=XML,basicstyle=\footnotesize]
<relationship>
  <relationshipType>served-by</relationshipType>
  <relatedResource ivo-id="ivo://org.gavo.dc/tap"
    >GAVO Data Center TAP service</relatedResource>
</relationship>
\end{lstlisting}

\item The declaration for the auxiliary capability, including the access
URL so clients to not need to follow the relationship just declared if
all they need is the access URL:
\begin{lstlisting}[language=XML,basicstyle=\footnotesize]
<capability standardID="ivo://ivoa.net/std/TAP#aux">
   <interface role="std" version="1.1" xsi:type="vs:ParamHTTP">
     <accessURL use="base">http://dc.zah.uni-heidelberg.de/tap</accessURL>
   </interface>
</capability>
\end{lstlisting}

\item Most importantly, the declaration of the table utype that lets
clients discover that this particular table contains EPNCore data:
\begin{lstlisting}[language=XML,basicstyle=\footnotesize]
<table>
  <name>toss.ivoa_lines</name>
  <title>TOSS</title>
  <description> The EPN-TAP 2.0 version of...</description>
  <utype>ivo://ivoa.net/std/linetap#table-1.0</utype>
  ...
</table>
\end{lstlisting}
\end{itemize}

That in the example record, the resource description is identical to the
description of the schema, which again is identical to the description
of the table is an artefact of LineTAP registrations being single-table
and is thus to be expected in most registrations of this type.


\subsection{Discovering LineTAP services}

LineTAP consumers in general are interested in TAP endpoints and table names for
lineTAP services.  By our registration pattern, this translates into
resources with TAP capabilities that have a standard key for version 1
LineTAP in a table utype.

Translated into RegTAP \citep{2019ivoa.spec.1011D}, the following query
would return TAP access URLs and the table names, using ADQL 2.1 CTEs
for readability:

\begin{lstlisting}[language=SQL]
SELECT DISTINCT table_name, access_url
FROM rr.res_table
  NATURAL JOIN rr.capability
  NATURAL JOIN rr.interface
WHERE
  table_utype='ivo://ivoa.net/std/linetap#table-1.%'
  AND standard_id LIKE 'ivo://ivoa.net/std/tap%'
  AND intf_role='std'
\end{lstlisting}

The \texttt{DISTINCT} in the main query is a rough filter that removes
entries duplicated because their tables are registred both in the main
TAP record and in an auxiliary capability.  

The regular expression in the utype match is to make sure minor version
increments do not prevent service discovery; by IVOA versioning rules,
all lineTAP services of minor version 1 can be operated by all lineTAP
clients of version 1.  We do not constrain the version of the TAP
service; clients may want to adapt the TAP discovery pattern to match
their specific needs.



\appendix
\section{Changes from Previous Versions}

No previous versions yet.
% these would be subsections "Changes from v. WD-..."
% Use itemize environments.


\bibliography{ivoatex/ivoabib,ivoatex/docrepo, localrefs}

\end{document}
