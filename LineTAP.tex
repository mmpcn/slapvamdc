\documentclass[11pt,a4paper]{ivoa}
\input tthdefs
\usepackage{float}
\usepackage{todonotes}

\lstloadlanguages{SQL,XML}
\lstset{flexiblecolumns=true,numberstyle=\small,showstringspaces=False,
  identifierstyle=\texttt}

\title{LineTAP: IVOA Relational Model for Spectral Lines}

% see ivoatexDoc for what group names to use here
\ivoagroup{???? group ????}

\author{Castro Neves, M.}
\author{Moreau, N.}
\author{Demleitner, M.}

\editor{Margarida Castro Neves, Nicolas Moreau}


\previousversion{This is the first public release}

\begin{document}
\begin{abstract}

This document proposes a set of  spectral line parameters that can be queried using the 
TAP protocol. Its purpose is to present a simpler way to query for spectral line data using 
VO Applications. Among the existing services, SSLDM/SLAP has very few services 
deployed and is not being used. On the other side, VAMDC is has plenty of services and 
spectral line information, but its data model is very complex. Therefore the need of a 
simpler model, using already established protocols like TAP for querying and retrieving the 
information. Besides the proposed set of parameters, a mapping from VAMDC data model 
has also been developed.

\end{abstract}


%\section*{Acknowledgments}


\section*{Conformance-related definitions}

The words ``MUST'', ``SHALL'', ``SHOULD'', ``MAY'', ``RECOMMENDED'', and
``OPTIONAL'' (in upper or lower case) used in this document are to be
interpreted as described in IETF standard RFC2119 \citep{std:RFC2119}.

The \emph{Virtual Observatory (VO)} is a
general term for a collection of federated resources that can be used
to conduct astronomical research, education, and outreach.
The \href{http://www.ivoa.net}{International
Virtual Observatory Alliance (IVOA)} is a global
collaboration of separately funded projects to develop standards and
infrastructure that enable VO applications.


\section{Introduction}

SLAP (Simple Line Access Protocol)\citep{2010ivoa.specQ1209O} is the VO standard for 
Spectral Line querying. SSLDM (Simple Spectral Line Data 
Model)\citep{2010ivoa.spec.1209O} 
is the VO data model for spectral lines.
Currently there are very few working services in the VO that use this protocols, meaning 
that the amount of spectral line data in the VO is small. 

On the other side, VAMDC services offer a great amount of spectral line data, and  so it's 
desirable to access this data in a VO-way. 
The queries to the VAMDC services might be simple, however, the data returned is much 
more than the necessary data for the common use cases in VO, besides the XSAMS 
\citep{XSAMS:Docs} output format being quite complex.

This document proposes LineTAP, a simple way to access spectral line data through a VO 
service through a relational data model. A selection of the relevant line quantities used in 
astronomy has been defined, which can be retrieved using TAP \citep{std:TAP}, which is 
an established IVOA protocol. 

To make use of the VAMDC data, a mapping from the from the 
\href{https://standards.vamdc.eu/#data-model}{VAMDC-XSAMS 
Data Model} to the relational model is 
also presented. Using TAP, the resulting data is returned in VOTABLE format,  easily 
readable by VO client applications. 




\section{Use Cases}

In most use cases, the goal is to search for spectral lines within a wavelength range:

\todo{ more use cases}

\begin{itemize}

\item selecting  spectral lines that fit the ones in the spectrum.
Normally when searching for lines within a wavelength range, a great quantity of lines are 
returned, with very similar wavelengths. It's necessary to select the one that really 
represents the transition that caused the line in the spectrum.
\item areas: Stars, galaxies, ISM, planets?
\item calculating redshift
\item comparing theoretical spectra to observed spectra
\item classify the star type according to spectral lines?
\item spectral lines from a specific atom (or molecule)
\item spectral lines from a specific ionisation level
\item spectral lines with specific (other) parameter
\end{itemize}

Another use case to consider is to get a list of all available chemical species 
from a service, as a first step before looking for spectral lines.

For any retrieved spectral line, the reference of the source article where it was published
should always be provided if available.



\section{Spectral Line Data}\label{quantities}

In the following table the selected quantities based on IVOA use cases are defined, 
including units, data types and mandatory status:

\todo{ method: define other method options besides experiment and theory?}

\begin{table}[H]
\small
%\begin{center}
\begin{tabular}{|l|l|l|p{0.8cm}|p{4cm}|}%{l l}% p{0.4\linewidth}}
\hline
\textbf{column name} & \textbf{unit} & \textbf{type} & \textbf{man da tory} & 
\textbf{description} \\
\hline
\hline
\texttt{title} & & string & & name of the species originating  the line.  \\
\hline
\texttt{spectral\_location} & J & double & yes & energy equivalent to wavelength 
in vacuum \\
\hline
\texttt{spectral\_error} & & double & & integrated error for the spectral location \\
\hline
\texttt{method} && string& & method the wavelength was obtained with:
\textit{experiment} (observed), or \textit{theory} (calculated from theoretical models) \\ 
\hline
\texttt{stoichiometric\_formula} & & string & &  the symbol of the chemical element or the 
molecule formula\\
\hline
\texttt{ion\_charge} & & integer & yes & ionisation level\\
\hline
\texttt{atom\_mass\_number} & & integer &  & atom mass number\\
\hline
\texttt{upper\_state\_configuration} & & string & & upper state configuration\\
\hline
\texttt{lower\_state\_configuration} & & string & & lower state configuration\\
\hline
\texttt{upper\_energy} & J  & double & & energy of the upper state \\
\hline
\texttt{lower\_energy} & J & double & & energy of the lower state\\
\hline
\texttt{inchi} & & string & yes & chemical species inchi \\
\hline
\texttt{inchikey} & & string & yes & chemical species inchikey \\
\hline
\texttt{einstein\_a} & & double  & &  Einstein A coefficient\\
\hline
\texttt{oscillator\_strength} & & double & & oscillator strength of radiative transition \\
\hline
\texttt{weighted\_oscillator\_strength} & & double  & &  Weighted oscillator strength of 
radiative transition \\
\hline
\texttt{line\_strength} & & string & & Total absorption by a spectra line \\
\hline
\texttt{line\_reference\_doi} & & string & & Digital Object Identifier of bibliography source \\
\hline
\texttt{line\_reference\_uri} & & string & & Web link to the publication\\
\hline
\end{tabular}

%\end{center}
\end{table}

\begin{itemize}
\item \texttt{spectral\_location} 
We use energy instead of wavelength to avoid having to provide extra information about
air/vacuum lengths. It is stored in the database in Joules (J). Different communities can use 
a provided  ADQL user-defined function to convert this energy to wavelength in the desired 
unit. They will be described in
\ref{sec:Protocol}.
\\
\item \texttt{spectral\_error} is the integrated error for the wavelength.
\todo {comment on symmetry/assimmetry of errors} 

\item \texttt{inchi} 
\item \texttt{inchiKey} 
\todo{ write about InChi, InChiKey, add Literature sources about it}
\end{itemize}


\section{Protocol }
\label{sec:Protocol}
\subsection{ Queries: LineTAP }

\subsection{User-defined functions}

\begin{itemize}
\item ivo\_spectral(value, unit)\\
Converts energy values in J to values in other units. Example:\\
\begin{quote}
SELECT ivo\_spectral(energy, 'Angstrom') AS lambda \\
 WHERE ivo\_spectral(energy, 'MHz') BETWEEN 88 AND 108
\end{quote}

\end{itemize}

\todo{ description, examples, translation VAMDC Queries}



\section{LineTAP use case examples}


\section{Mapping from VAMDCXSAMS}

The quantities used in the mapping belong to the following branches of VAMDCXSAMS:\\
\\
\textit{XSAMSData.Species.Atoms.Atom}  referred as ATOM,\\
\textit{XSAMSData.Species.Atoms.AtomicState}  referred as ATOMICSTATE,\\
\textit{XSAMSData.Species.Molecules.Molecule}, referred as MOLECULE,\\
\textit{XSAMSData.Processes.Radiative.RadiativeTransition}, referred as RADTRANS.\\
\textit{XSAMSData.Sources}, referred as RADTRANS.\\


The VAMDC-TAP query language defines a list of keywords, called 
\href{https://standards.vamdc.eu/dictionary/returnables.html}{Returnables}, 
listing the quantities that 
can be returned by a service.They correspond to an element in the XSAMS data model. 
The returnables supported by a service are declared by a service manager when he 
deploys the software on its database, and saved in the VAMDC registry. \\

Listed below are the mappings for the atomic quantities defined in \ref{quantities} (at the 
moment only atoms, not molecules):

\renewcommand{\descriptionlabel}[1]{\hspace{\labelsep}\texttt{#1}}
\begin{description}

\item [spectral\_location] (in J)\hfill\\
    \textit{data model:} RADTRANS.EnergyWavelength\\
	\textit{returnables needed:} RadTransWavelength, RadTransWavelength.Vacuum, 
RadTransWavelengthAirToVac, RadTransWavelengthUnit\\
	\textit{constraints:} RadTransWavelengthVacuum = true; else use 
RadTransWavelengthAirToVac. Original Value in $Angstrom$ has to be converted to $J$\\
	
\item [spectral\_error]  \hfill\\\todo{Wavelength error? How to get it from VAMDC data?}

    \textit{data model:} RADTRANS.EnergyWavelength\\
	\textit{returnables needed:} RadTransWavelength, RadTransWavelength.Vacuum, 
RadTransWavelengthAirToVac, RadTransWavelengthUnit\\
	\textit{constraints:} RadTransWavelengthVacuum = true; else use 
RadTransWavelengthAirToVac.\\
	
\item [method] \hfill\\
	\textit{data model:} Methods.Method.Category\\
	\textit{returnables needed:} RadTransWavelengthMethod\\
	\textit{constraints:} 
Methods.Method.MethodID=RadTransWavelength.Wavelength@methodRef\\

\item [inchi] \hfill\\
	\textit{data model:} ATOM.Isotope.Ion.InChi,  
MOLECULE.MolecularChemicalSpecies.InChI\\
        \textit{returnables needed:} AtomInchi, MoleculeInChi\\
	\textit{constraints:} RadTrans.SpeciesRef = Atom.SpeciesID or MoleculeSpeciesID
	
\item [inchikey] \hfill\\
	\textit{data model:} ATOM.Isotope.Ion.InChiKey, 
MOLECULE.MolecularChemicalSpecies.InChIKey \\
         \textit{returnables needed:} AtomInchiKey, MoleculeInChiKey\\
         \textit{constraints:} RadTrans.SpeciesRef = Atom.SpeciesID or  Molecule.SpeciesID

\item [stoichiometric\_formula] \hfill\\
	\textit{data model:} ATOM.ChemicalElement.ElementSymbol,\\ MOLECULE. 
MolecularChemicalSpecies.StoichiometricFormula\\
	\textit{returnables needed:} RadTrans.SpeciesRef, Atom.SpeciesID, 
Molecule.SpeciesID, Atom.Symbol, Molecule.MoleculeStoichiometricFormula\\
	\textit{constraints:}  RadTrans.SpeciesRef = Atom.SpeciesID or  Molecule.SpeciesID

\item [ion\_charge]\hfill\\
	\textit{data model:} ATOM.Isotope.Ion.IonCharge or 
MOLECULE.MolecularChemicalSpecies.IonCharge \\
	\textit{returnables needed:} RadTransSpeciesRef, AtomSpeciesID, AtomIonCharge, 
MoleculeSpeciesID, MoleculeIonCharge \\
	\textit{constraints:}RadTransSpeciesRef=AtomSpeciesID or  Molecule.SpeciesID

	\item [atom\_mass\_number]\hfill\\
	\textit{data model:} ATOM.Isotope.IsotopParameters.MassNumber \\
	\textit{returnables needed:} AtomMassNumber\\
	\textit{constraints:}RadTransSpeciesRef=AtomSpeciesID

	\item [upper\_state\_configuration]\hfill\\
	\textit{data model:} ATOMICSTATE.Description, MOLECULARSTATE.Description\\
	\textit{returnables needed:} RadTransUpperStateRef, AtomStateID, 
MoleculeStateID,\\ 
AtomStateDescription, MoleculeStateDescription\\
	\textit{constraints:}  RadTransUpperStateRef = AtomStateID

	\item [lower\_state\_configuration]\hfill\\
	\textit{data model:} ATOMICSTATE.Description, MOLECULARSTATE.Description\\
	\textit{returnables needed:} RadTransUpperStateRef, AtomStateID, 
MoleculeStateID,\\ 
AtomStateDescription, MoleculeStateDescription\\
	\textit{constraints:}  RadTransUpperStateRef = AtomStateID or MolecularStateID
	
	\item [upper\_state\_energy]\hfill\\
	\textit{data model:} ATOMICSTATE.AtomicNumericalData.StateEnergy, \\ 
MOLECULARSTATE.MolecularStateCharacterisation.StateEnergy\\
	\textit{returnables needed:} RadTransUpperStateRef, AtomStateID, 
MoleculeStateID\\
	\textit{constraints:} 
 RadTransUpperStateRef = AtomStateID or MoleculeStateID. 
 Original value in $cm^{-1}$ has to be converted to $J$
	
	\item [lower\_state\_energy]\hfill\\
	\textit{data model:} ATOMICSTATE.AtomicNumericalData.StateEnergy, \\ 
MOLECULARSTATE.MolecularStateCharacterisation.StateEnergy\\
	\textit{returnables needed:} RadTransLowerStateRef, AtomStateID,  
MoleculeStateID\\
	\textit{constraints:}  RadTransLowerStateRef = AtomStateID or MoleculeStateID.
	Original value in $cm^{-1}$ has to be converted to $J$

	\item [einstein\_a]\hfill\\
	\textit{data model:}  RADTRANS.Probability.TransitionProbabilityA\\
	\textit{returnables needed:} RadTransProbabilityA\\
	\textit{constraints:}

	\item [oscillator\_strength]\hfill\\
	\textit{data model:}  RADTRANS.Probability.TransitionOscillatorStrength\\
	\textit{returnables needed:} RadTransProbabilityOscillatorStrength\\
        \textit{constraints:}

	\item [weighted\_oscillator\_strength]\hfill\\
	\textit{data model:}  RADTRANS.Probability.TransitionWeightedOscillatorStrength\\
	\textit{returnables needed:} RadTransProbabilityWeightedOscillatorStrength\\
 	   \textit{constraints:}

	\item [line\_reference\_doi]\hfill\\
	\textit{data model:} SOURCES.Source.DigitalObjectIdentifier\\
	\textit{returnables needed:} SourceDOI\\
	\textit{constraints:}

	\item [line\_reference\_uri]\hfill\\
	\textit{data model:} SOURCES.Source.UniformResourceIdentifier\\
	\textit{returnables needed:} SourceURI\\
	\textit{constraints:}

	\item [line\_strength]\hfill\\
	\textit{data model:} RadTrans.Probability.TransitionLineStrength\\
	\textit{returnables needed:} RadTransProbabilityLineStrength\\
        \textit{constraints:}

\item [title]\hfill\\
	a string composed by the values of  \texttt{element\_symbol} and \texttt{ion\_charge}


\end{description}

\section{LineTAP and the VO Registry}

\subsection{Registering LineTAP-conforming Tables}

LineTAP tables are registered using VODataService
\citep{2010ivoa.spec.1202P} tablesets, where the table utype is set to
$$\hbox{\verb|ivo://ivoa.net/std/linetap#table-1.0|}.$$

The tableset is normally contained in a VODataService \xmlel{CatalogService}
record with a TAP capability, and this capability normally is an auxiliary
capability as per DDC \citep{2019ivoa.rept.0520D}.  For one-table
services a full TAPRegExt \citep{2012ivoa.spec.0827D} capability is also
allowed; other resource types can be used for registration as
appropriate.

Further capabilities, for instance for full VAMDC or legacy SLAP
services, may be given in the same record.

An example for a registry record in VOResource, for the case of
using an auxiliary capability referencing a main TAP service comes with
this document\footnote{\auxiliaryurl{example-record.xml}}.

The noteworthy points in the record are:

\begin{itemize}
\item A \xmlel{relationship} element referencing the main TAP service 
through which the service is queriable as per DDC:
\begin{lstlisting}[language=XML,basicstyle=\footnotesize]
<relationship>
  <relationshipType>served-by</relationshipType>
  <relatedResource ivo-id="ivo://org.gavo.dc/tap"
    >GAVO Data Center TAP service</relatedResource>
</relationship>
\end{lstlisting}

\item The declaration for the auxiliary capability, including the access
URL so clients to not need to follow the relationship just declared if
all they need is the access URL:
\begin{lstlisting}[language=XML,basicstyle=\footnotesize]
<capability standardID="ivo://ivoa.net/std/TAP#aux">
   <interface role="std" version="1.1" xsi:type="vs:ParamHTTP">
     <accessURL use="base">http://dc.zah.uni-heidelberg.de/tap</accessURL>
   </interface>
</capability>
\end{lstlisting}

\item Most importantly, the declaration of the table utype that lets
clients discover that this particular table contains EPNCore data:
\begin{lstlisting}[language=XML,basicstyle=\footnotesize]
<table>
  <name>toss.ivoa_lines</name>
  <title>TOSS</title>
  <description> The EPN-TAP 2.0 version of...</description>
  <utype>ivo://ivoa.net/std/linetap#table-1.0</utype>
  ...
</table>
\end{lstlisting}
\end{itemize}

That in the example record, the resource description is identical to the
description of the schema, which again is identical to the description
of the table is an artefact of LineTAP registrations being single-table
and is thus to be expected in most registrations of this type.


\subsection{Discovering LineTAP services}

LineTAP consumers in general are interested in TAP endpoints and table names for
lineTAP services.  By our registration pattern, this translates into
resources with TAP capabilities that have a standard key for version 1
LineTAP in a table utype.

Translated into RegTAP \citep{2019ivoa.spec.1011D}, the following query
would return TAP access URLs and the table names, using ADQL 2.1 CTEs
for readability:

\begin{lstlisting}[language=SQL]
SELECT DISTINCT table_name, access_url
FROM rr.res_table
  NATURAL JOIN rr.capability
  NATURAL JOIN rr.interface
WHERE
  table_utype='ivo://ivoa.net/std/linetap#table-1.%'
  AND standard_id LIKE 'ivo://ivoa.net/std/tap%'
  AND intf_role='std'
\end{lstlisting}

The \texttt{DISTINCT} in the main query is a rough filter that removes
entries duplicated because their tables are registred both in the main
TAP record and in an auxiliary capability.  

The regular expression in the utype match is to make sure minor version
increments do not prevent service discovery; by IVOA versioning rules,
all lineTAP services of minor version 1 can be operated by all lineTAP
clients of version 1.  We do not constrain the version of the TAP
service; clients may want to adapt the TAP discovery pattern to match
their specific needs.



\appendix
\section{Changes from Previous Versions}

No previous versions yet.
% these would be subsections "Changes from v. WD-..."
% Use itemize environments.


\bibliography{ivoatex/ivoabib,ivoatex/docrepo, localrefs}

\end{document}
